\documentclass[a4paper,oneside]{article}
\usepackage[english]{babel}
\usepackage[utf8]{inputenc}
\usepackage{graphicx}
\usepackage{amsfonts}
\usepackage[affil-it]{authblk}
\usepackage[table]{xcolor}
\usepackage{pagecolor}
\usepackage{afterpage}
\usepackage{changepage}
\usepackage{gitinfo2}

\usepackage[a4paper, includeheadfoot]{geometry}
\geometry{a4paper, total={210mm,297mm}, left=38mm, right=38mm, top=24mm, bottom=48mm}

\usepackage{fancyhdr}
\pagestyle{fancy}

\usepackage{xcolor}
\graphicspath{ {./images/} }
\usepackage[pdftex,
    pdfauthor={Evan Green \& Adam Waldenberg, The Unigrid Foundation},
    pdftitle={Unigrid: The Next Internet Revolution},
    pdfsubject={Orange Paper},
    pdfkeywords={blockchain;internet;bitcoin;unigrid;sharding;segmentation;consensus;decentralized;governance;gridnode},
    pdfproducer={Latex with hyperref, or other system},
    pdfcreator={pdflatex, or other tool}]{hyperref}

\let\oldhref\href
\renewcommand{\href}[2]{\oldhref{#1}{\bfseries#2}}
\definecolor{txtcolor}{rgb}{0.88,0.88,0.88}
\hypersetup{colorlinks = true, urlcolor = txtcolor, citecolor = orange, linkcolor = txtcolor}

\author{\textit{Evan Green, \href{mailto:evan@unigrid.org}{evan@unigrid.org}}\\
\textit{Adam Waldenberg, \href{mailto:adam@unigrid.org}{adam@unigrid.org}}}
\affil{The Unigrid Foundation}

\title{
	\makebox[\textwidth]{\hspace{600pt}\tikz \fill[orange] (18,1.4) rectangle (0,0);}
	\vspace{60pt}
	\begin{center}
		\includesvg[height=80pt]{unigrid-wide}
	\end{center}
	\vspace{35pt}
	\textbf{The Next Internet Revolution}
	\vspace{10pt}
}
\date{\emph{Version \gitRel\hspace{5pt}(\gitCommitterDate)}}

\definecolor{headerbg}{rgb}{0,0,0.18}
\definecolor{headerbgl}{rgb}{0.07,0.07,0.24}

\usepackage[framemethod=tikz]{mdframed}
\usepackage[inkscapearea=page]{svg}

\mdfdefinestyle{unigridheader}{
	userdefinedwidth=381pt,
	backgroundcolor=headerbg,
	linecolor=orange,
	linewidth=3pt,
	leftline=false,
	rightline=false,
	bottomline=true,
	topline=false,
	innerbottommargin=0,
	innerrightmargin=0
}

\mdfdefinestyle{textimage}{
	backgroundcolor=headerbg,
	linecolor=orange,
	linewidth=3pt,
	leftline=false,
	rightline=false,
	bottomline=true,
	topline=true,
	innerrightmargin=0,
	innerleftmargin=0,
	innertopmargin=4pt,
	innerbottommargin=4pt,
	leftmargin=0
}

\setlength{\headheight}{70pt}%
\renewcommand{\headrulewidth}{0pt}
\renewcommand{\footrulewidth}{0pt}
\lhead{
	\begin{mdframed}[style=unigridheader]
		\begin{flushright}
			\includesvg[height=30pt]{unigrid-wide}
		\end{flushright}
	\end{mdframed}
}

\renewcommand{\familydefault}{\sfdefault}
\color{txtcolor}

\begin{document}

\newpagecolor{headerbg}
\clearpage\maketitle
\thispagestyle{empty}
\newpage
\vspace*{+34pt}
\begin{abstract}
\noindent The original Internet was envisioned to become an open and distributed network that was scalable and fair, allowing access to data and services without surveillance or security concerns. However, in recent years, the Internet has become increasingly centralized, controlled and monopolized by big businesses running huge data centers. This centralization has given big entities and businesses unprecedented control of the traffic and data of the network.

\vspace{0.1cm}
\begin{mdframed}[style=textimage]
	\includegraphics[width=331pt]{lights}
\end{mdframed}

\noindent With a majority of resources on the Internet owned by a minority of big corporations, these businesses are able to maintain a high cost to value ratio. The centralization and lack of fault tolerance and redundancy makes your personal data less secured. As a remedy to this deteriorating trend, we suggest the inception of a decentralized and consensus-driven segmented blockchain network based on a striped storage solution. The protocol allows for a completely decentralized and secure blockchain-based Internet where any user, including private persons, can host an income-generating service node, aiding the network with compute cycles, bandwidth, and storage space. To allow for complete utilization of the network, an access layer is provided, allowing for the development of protocols, services, and infrastructure.
\vspace*{+40pt}
\end{abstract}

\newpage
\section{Problem}
The current Internet is dominated by large multi-national conglomerates. Businesses like Google, Microsoft, and Amazon take up vast segments of the total market share. Amazon Web Services (AWS) has a dominate position, having nearly 50\% of that total market share \cite{jeb2019}.

\begin{center}
\includegraphics[width=45pt]{centralized}
\hspace{1.5cm}
\includegraphics[width=45pt]{unreliable}
\hspace{1.5cm}
\includegraphics[width=45pt]{unsecure}
\\
\vspace{0.1cm}
\hspace{0pt}\emph{Centralized}\hspace{50pt}\emph{Unreliable}\hspace{54pt}\emph{Unsecure}\hspace{24pt}
\end{center}

\noindent A near majority of all Internet data is stored on servers owned by a single entity \cite{jeb2019}. As a consequence, this results in much of the current traffic being routed through Amazon’s servers and data centers \cite{jeb2019}. Because of this, important routes often get overloaded during peak hours, slowing down services - resulting in a degraded user experience. In fact, routes regularly become so overloaded that it effectively causes servers on the other end of the communication queue to be unreachable - something most Internet users have experienced. If your provider goes offline or gets disrupted - your service and data becomes unreachable.

\vspace{0.05cm}
\begin{mdframed}[style=textimage]
	\includegraphics[width=381pt]{puzzle}
\end{mdframed}

\noindent The centralized nature of the resource ownership on the Internet also results in under-utilization of servers, with idle CPU cycles and gigabytes or potentially terabytes of unused storage space on a single server. This is especially true when businesses choose to cost-optimize their operations and rely on their own dedicated servers, rather than relying on the more expensive cloud provider solutions.

In some countries, Internet censorship is so prevalent that the public is limited to what news resources they are allowed to access and view \cite{wiki2021}. The original intention of the Internet was to create an open and globally accessible network, allowing everybody with access to view and take part of that information, regardless of their originating location.

For privacy, cost and security reasons, this situation is not sustainable. Hackers, oppressive governments and monopolizing businesses continuously develop better tools that infringe on your data and privacy - creating a very vulnerable situation.

\begin{center}
	\includegraphics{adam-evan}
\end{center}

\section{Team}
Unigrid was the brainchild of Adam Waldenberg and Evan Green, luminaries in the fields of software development, enterprise applications, and marketing. With a cumulative experience of over /50 years in programming, their portfolio is as vast as it is varied, spanning from intricate enterprise applications to immersive games and user-friendly frontends.

Adam, serving as the CTO of Unigrid, brings a unique blend of academic and practical prowess to the table. With over a decade spent imparting knowledge and tutoring at Chalmers University of Technology, he has shaped the minds of many budding tech enthusiasts. Further solidifying his reputation in the tech community, Adam has been at the helm of a data center nestled in the heart of Gothenburg since late 2015, showcasing his commitment to infrastructure and the backend intricacies that are so pivotal to the digital world.

Evan Green, the CEO of Unigrid, has achievements that are no less remarkable. Beyond his expertise in software development, he has masterfully led marketing campaigns for some of the world's top Fortune 500 companies. As the chief executive, he steers the overall direction and strategy of Unigrid, ensuring its growth and stability. His experience gives him a unique perspective, blending technical acumen with a deep understanding of market dynamics and consumer behavior.

Yet, the strength of Unigrid doesn't solely rest on these two visionaries. The project is bolstered by a diverse and talented team comprising developers, strategists, and experts from various domains. Partner companies like ValueX \cite{valuex2023} and NFMedia \cite{nfmedia2023} further elevate Unigrid's capabilities. These collaborations bring invaluable insights into business decisions and aid in the multifaceted development processes. It's this collective effort, led by CEO Evan Green and CTO Adam Waldenberg but supported by a wide-ranging team, that positions Unigrid for a bright and innovative future.


\section{Discussion}
The primary goal of Unigrid is to implement an anonymous and decentralized communication solution that offers data storage and compute cycle utilization. Private individuals and businesses that value privacy or require secure data storage can take advantage of this feature.

As an example, sensitive documents such as medical records, can be stored on-chain and only shared with selected individuals or entities. The data can be shared using a key that has an expiration timer, meaning that the sharing party never has to worry about the risk that the data will be exposed or accessed by an unwanted party. It should be the owning individual or business that controls how their private data is accessed and shared. It should never be in the control of a third party - something that is usually the case today. When the data is controlled by a third party, the risk of information leaks increases dramatically. Sometimes data is stolen by hackers, other times it is leaked or shared on purpose (in secrecy or otherwise). Whatever the reason - the Unigrid network gives the control back to the user or the party that actually owns the data.

\begin{center}
\includegraphics[width=45pt]{anonymous}
\hspace{1.5cm}
\includegraphics[width=45pt]{scalability}
\hspace{1.5cm}
\includegraphics[width=45pt]{security}
\\
\vspace{0.11cm}
\hspace{0pt}\emph{Anonymity}\hspace{49pt}\emph{Scalability}\hspace{58pt}\emph{Security}\hspace{25pt}
\end{center}

\noindent The Internet of Things (IoT) market is undergoing rapid growth, introducing a wealth of opportunities for innovation and improved connectivity. However, this expansion is not without its challenges. A pressing issue in the IoT realm is ensuring the security and integrity of data as devices, often from different manufacturers and utilizing distinct protocols, exchange information. Traditional security models sometimes struggle to offer foolproof solutions in such a complex environment.

Blockchain technology emerges as a beacon in this scenario. Within a blockchain framework, the ledger is meticulously maintained to be secure, with each entry being resistant to unauthorized alterations. This decentralized approach effectively removes the conventional need for absolute trust between communicating devices. When IoT systems integrate with a blockchain network, such as Unigrid, they inherit an augmented layer of security, ensuring that the data not only remains trustworthy but also significantly less vulnerable to potential breaches.

\noindent The sheer volume of devices connecting to IoT presents another challenge: scalability. It's not just about preserving the integrity of data; the network must also support vast numbers of devices without compromising performance or speed. A surge in the number of IoT devices means the infrastructure must adapt quickly and efficiently. As noted by Shruti Jain from Deloitte \cite{jain2021}, blockchain and distributed ledger technology (DLT) present promising avenues to tackle these IoT security and scalability issues, ensuring a more interconnected and secure future.

\begin{mdframed}[style=textimage]
	\includegraphics[width=381pt]{communication}
\end{mdframed}

\subsection{Gridnodes}
A gridnode is similar in function to a masternode as initially conceived by the Darkcoin project, where anybody can set-up a node on a server by locking up a certain amount of tokens to that node. These masternodes then provide a service to the network by contributing network bandwidth in the form of helping other nodes to synchronize and verify blocks on the network. Because these masternodes secure the network in this way, they are rewarded in tokens.

The Unigrid takes this idea further and implements gridnodes. Gridnodes are scored by the quality of the service they provide to the network and its nodes. For example, latency, the amount of available storage capacity, processing speed and network bandwidth are all taken into account. Each gridnode is then scored based on how they rank compared to other gridnodes.

With a regular masternode, reward frequency and occurrence is based on a naively calculated list. The network simply stores this list into memory and will cycle through each block to select the next winner. With the Unigrid network and gridnodes, scoring and rewards work differently. For example, when a request is made to store data on the network, there will be a check performed - looking for the next gridnode candidate. This check will see which gridnode has received the last reward plus scan the scores of each gridnode. Once a winner is selected, this gridnode will collect the data submitted and work with the other gridnodes to shard it across the network. This sharding allows the network to scale efficiently.

\subsection{Side Chains}
The Unigrid network places blocks in appropriate side chains based on the type of work currently being sent by a communicating node. Each side chain is controlled by a group of gridnodes. A side chain is a specialized blockchain with very specific characteristics, optimized for the type of work it is meant for. For example, data storage and compute cycles don't need the brute speed that a direct communication channel requires. Therefore, the Unigrid network treats these workloads differently. The communication side chains are smaller and kept in-memory, using a very fast but less collision resistant hashing algorithm. The blockchains that handle compute cycles and storage, on the other hand, are more resistant to hash collisions - making them very secure. However, they can not handle the same throughput as the in-memory chains using the less resistant hashing variant.

\begin{center}
\vspace{0.1cm}
\includegraphics[width=45pt]{efficiency}
\hspace{1.5cm}
\includegraphics[width=45pt]{segmented}
\\
\vspace{0.1cm}
\hspace{10pt}\emph{Efficiency}\hspace{46pt}\emph{Segmentation}
\end{center}

\noindent The Unigrid network can create side chains on demand. The network continously rebalances, with blockchains being created, removed and restructured as the members of them and their data changes. This is possible because the network uses parity blocks and thus has the ability to rebuild side chains and throw away stale blocks.

\subsection{Data Storage}
The design of the Unigrid network allows it to store large amounts of data on-chain that can be accessed at any time. The data stored on the network is either public or private. Public data is viewable by anybody with access to the network. Private data, on the other hand, will only be accessible via a private key and has to be decrypted in order to allow access to the data.

\noindent Cloud data storage is projected to reach a market size of \$297.54 billion by 2027 \cite{fort2021} and might even grow bigger if networks such as Unigrid reach big general adoption. Non-Fungible Tokens (NFTs) are tokens where only one specific token can be in existence at a given time. These allow for proof of ownership of a digital asset. The problem with the current system is that the only thing being stored on-chain is the actual proof of ownership. The asset itself that an NFT is tethered to is actually stored on a normal web server or some other means of storage. Consequently, if that web server or storage shuts down or is moved, the data for that NFT is lost.

\vspace{0.2cm}
\begin{mdframed}[style=textimage]
	\includegraphics[width=381pt]{hard-drive}
\end{mdframed}

\noindent The Unigrid network, is a ideal solution for handling NFTs and digital assets. Thanks to the redundancy, fault-tolerance and side chains of the network, the actual asset data can be stored in a way that ensures it can never be removed or lost by accident. This provides more long-term solution to where assets, that in some cases cost millions of dollars, are permanently and securely stored.

The NFT market has exploded in the past year. According to Joseph Young of Forbes, the market cap has grown an astounding 1758\% \cite{young2021}.

\subsection{Sharding \& Parity Blocks}
In a typical blockchain, data is stored in a vertical structure with each new block being appended to the previous block. However, this is not an ideal solution when you want to create a low latency network capable of fast data access and transfers. The solution to solve this problem on the Unigrid network involves sharding data across multiple blockchains. Furthermore, parity blocks are introduced to achieve redundancy and fault tolerance. The result is a network built for speed, security and scalability.

\begin{center}
\vspace{0.1cm}
\includegraphics[width=45pt]{redundancy}
\hspace{1.5cm}
\includegraphics[width=45pt]{tolerance}
\hspace{1.5cm}
\includegraphics[width=45pt]{load-balance}
\\
\vspace{0.12cm}
\hspace{8pt}\emph{Redundancy}\hspace{38pt}\emph{Fault-Tolerance}\hspace{30pt}\emph{Load Balancing}
\end{center}

\noindent Sharding is a database architecture pattern related to horizontal partitioning - the practice of separating one table’s rows into multiple different tables, known as partitions \cite{mark2019}. Data on the Unigrid network will be sharded across nodes into group swarms called shard groups. This allows the network to become a self-replicating CDN (Content Delivery Network) where data is often pulled from the most optimal location. This minimizes the network latency currently experienced on the modern web where you may be pulling data from a webpage in one centrally installed location.

A video streaming service would be able to take advantage of the network and the available speeds and fault-tolerance. As the videos stored on the network would be sharded and spread out on the network, users using the service would always load content from the fastest gridnode swarm with the lowest latency. The implementation needed from the actual video service would be simpler and could rely on the built in redundancy and fault-tolerance of the network, meaning the time and resources needed to implement such as service would also be far lower when compared to a more traditional solution.

\subsection{Compute Cycles}
As the network grows and more gridnodes come online, the compute power available on the network will continuously increase. The scoring algorithms on the network will encourage operators to run gridnodes with a lot of resources - resulting in a network with a constantly increasing storage capacity and available computational power. Using Unigrid tokens, organizations and users will be able to rent these resources and use them for custom workloads and data.

\vspace{0.12cm}
\begin{mdframed}[style=textimage]
	\includegraphics[width=381pt]{compute}
\end{mdframed}

\noindent One use case appropriate for this would be a scientific study that needs to run some very complex computationally intensive tasks. In return, the gridnodes offering the service would be awarded a certain amount of Unigrid, as decided by the network. 

Another way to take advantage of the computational power on the network would be to run a container or VPS (Virtual Private Server) on top of the shard groups. Thanks to the redundancy and parity blocks - the network allows for the deployment of containers and server instances that never go offline unintentionally.

In future milestones, the Unigrid network will also support GPU workloads. This means that cycles specifically written for graphical processors will also be handled by the gridnodes on the network. As an even more distant goal and when the available technology allows for a working implementation, the foundation also plans to add support for quantum cycles - which would allow the network to process work specifically designed for quantum processors.

\subsection{Migration}
Connecting directly to the ordinary Internet exposes your personal information to external parties. In order to avoid this leaking of personal data, many Internet users use a VPN (Virtual Private Network). This helps to protect your personal data by creating an encrypted tunnel to the Internet. However, the question is - can you fully trust the VPS provider?

\vspace{0.05cm}
\begin{mdframed}[style=textimage]
	\includegraphics[width=381pt]{migration}
\end{mdframed}

\noindent When accessing the ordinary Internet via the Unigrid network, a connection is made via a locally running proxy server, assigning a fingerprint to a specific communication channel. A gridnode will pick up this fingerprint and act as the “outlet” for that connection. The data being sent is encrypted. The gridnodes themselves act as a swarm of proxies - meaning nobody actually knows who picks up the data or who is the originating party. This is very different from TOR (onion routing), which sends data via a chain of hops.

The anonymous communication layer is one of the first milestones planned to be developed. This promotes the network recognition and offers a very important use case for new users.

\subsection{Governance}
Control of the network itself and deciding what is allowed and not is handled by the gridnodes and the nodes running on the network. A voting system similar to the governance system in other cryptocurrency projects will be used to vote on important decisions and spork changes. As soon as the foundation changes certain settings on the network, the gridnodes on the network have a choice to either acceppt or reject the change. These settings allow the network to control its own behaviour. The foundation can add blocks that should be blacklisted on the network and change a lot of different settings that changes the way the different implementations and algorithms used on the network actually behave.

\vspace{0.05cm}
\begin{mdframed}[style=textimage]
	\includegraphics[width=381pt]{abstract-particles}
\end{mdframed}

\noindent Any major updates to the network itself will also be voted on by the gridnodes. This creates a fully decentralized and democratic network that is controlled by its users - not by a central governing body.

\subsection{Past Decentralization Attempts}
Numerous projects in the digital landscape champion the notion of decentralization, promoting it as a key feature of their networks. However, upon closer examination, it becomes evident that the majority have limitations that deviate from the core principles of true decentralization. Many such projects necessitate the use of specialized hardware, creating barriers to entry for the average user. Furthermore, instead of promoting an open and inclusive environment, they restrict participation by implementing systems like written applications and rigorous vetting processes.

This approach does not genuinely eliminate centralization; it merely shifts it. Rather than distributing power and control across the network, they simply relocate the centralization to a different point, inadvertently introducing new potential vulnerabilities and points of failure. This contradicts the essence of decentralization, which is to prevent the concentration of power and to eliminate single points of failure.

Unigrid stands out by taking a fundamentally different approach. It envisions a system where barriers are minimized, and inclusivity is maximized. With Unigrid, participation isn't just a possibility—it's an invitation. Anyone, regardless of their technical background or resources, can join the network and take on the role of a gridnode host. This not only broadens the base of participants but also strengthens the network's resilience and reach. More importantly, it genuinely democratizes the system. Every gridnode is empowered with a voice, allowing them to actively participate in decision-making by voting on various proposals and network changes. This ensures that the network evolves in a manner that is in the best interest of its entire community, not just a select few.

\subsection{Market Adoption}
To ensure the financial stability of the the Unigrid Foundation, some departments within the foundation will focus on developing services around or for the Unigrid network. One part of the foundation will work towards developing and providing combined hardware and software solutions that corporations can buy in order to easily get access and contribute to the network. A second part of the foundation will form a legal division. This division will work with the daily legal issues within the foundation but also work towards the goal of supporting start-ups who want to work within the blockchain market. With access to a private data center readily available, the foundation will, as a third service, have the ability to help users contribute to the network by offering data center space for those who can not run gridnodes at home or at their own company. An open VPN service will also be created on the network to gain traction. 

\vspace{0.12cm}
\begin{mdframed}[style=textimage]
	\includegraphics[width=381pt]{road}
\end{mdframed}

\noindent The Unigrid Foundation will also be working to partner with universities and scientific research organizations. The foundation will allocate a certain percentage of compute cycles and data storage space to these institutions. With many years of teaching experience at a university level available within the foundation, education will be one of the goals with these partnerships. With more knowledge about blockchain at the universities, more students will be able to contribute to the blockchain community as a whole and to the Unigrid network. Since the foundation is based in Gothenburg, close to Chalmers University of Technology, students will also be offered to write their master thesis at the Unigrid Foundation. There are also plans to create grant programs which can be applied to, allowing access to funds from the foundation to further the development and adoption of the network. 

The foundation plan to organize hackathons and similar events to promote growth and development on the Unigrid network. The community will be one of the foundations biggest assets, somewhere where the foundation can look for new hires among the many eager developers and blockchain enthusiasts within the community.

\subsection{Competitive Advantage}
The Unigrid network will keep the cost of data storage and data processing to a minimum. With gridnodes constantly competing for work, costs will be drastically lower than alternative services.

\noindent Data on the network is automatically backed up and spread across the network for redundancy. If one gridnode goes down, its data is accessible via other nodes in the shard group. Consequently, the data is always online and accessible. The network itself will never go offline, allowing continuous and uninterrupted access to your data and apps.

\subsection{Foundational Structure}
The foundation will be overseen by the county administrative board of the region, helping to ensure that assets are properly managed and that the foundation charter is adhered to.

\vspace{0.05cm}
\begin{mdframed}[style=textimage]
	\includegraphics[width=381pt]{foundation}
\end{mdframed}

\noindent Developing a public utility under a Swedish foundation that is socially beneficial grants several tax benefits and allows the foundation to minimize tax expenditures, increasing funding for internal development and research sponsorship into the fields of distributed communication and storage.

Investments into high-yield assets with additional returns from the Unigrid network will allow the foundation to be self-funding after the initial public sales.

\subsection{Sale Participation \& Early Token Holders}
The Unigrid project, a visionary undertaking in the decentralized landscape, is currently in its early stages, laying the groundwork for what could be a transformative initiative. As it navigates the initial challenges and milestones, The Unigrid Foundation plays a crucial role in its trajectory. This recently established organization, formed with a vision to foster transparency and ethical practices, is at the helm of all token sales, ensuring that they are conducted with integrity and efficiency.

The true essence of the Unigrid project lies not just in its technology but in its approach to decision-making. Rather than relying on a centralized few, the project embraces the principles of decentralization in its governance. Which partners to collaborate with, and how tokens are allocated, are critical choices that shape the project's future. Remarkably, these decisions are not taken behind closed doors. Instead, they are decided by the community, the very heart of Unigrid. Through governance votes, every participant has a voice, reflecting the project's commitment to collective wisdom and inclusive decision-making.

By integrating a blend of technological innovation with democratic governance the network exemplifies a forward-thinking approach, setting a standard for other decentralized endeavours to follow.

\newpage
\section{Tokenomics}
Throughout the evolution of Unigrid, our tokenomics have undergone meticulous refinement across several iterations. This continuous improvement process has been enhanced by our collaboration with trusted partners, ValueX and NFMedia. Their insights and expertise have played a pivotal role in shaping a robust and resilient tokenomic framework.

The primary objective behind our tokenomics has always been to establish a resilient ecosystem. Protecting the market is a top priority, ensuring a stable environment that is resistant to manipulative practices and unforeseen market disruptions.

The value of our token isn't solely monetary; it represents the utility and functionality within the Unigrid system. By developing a tokenomic structure that promotes stability, we aim to ensure that the token remains consistently reliable in fulfilling its intended purpose within our network.

\begin{center}
	\includesvg[height=250pt]{tokenomics}
\end{center}

Furthermore, our gridnode operators are essential pillars of the Unigrid ecosystem. Their continuous efforts and contributions are invaluable, and our tokenomics are thus designed to recognize and reward their commitment. This approach ensures they are adequately incentivized and that their interests align with the overall health and growth of the network.

\noindent In essence, through iterative design and with the collaboration of partners like ValueX and NFMedia, our tokenomics have been crafted to prioritize the stability and well-being of the market, the utility of the token, and the prosperity of the gridnode operators.

\section{Milestones \& Roadmap}
The Unigrid Foundation, with a vision for a seamless and efficient future network, has laid out a roadmap comprising several milestones. These milestones serve as guiding markers, providing direction and clarity for the network's development journey. The foundation is not just focused on creation; it's equally committed to the network's ongoing evolution. By maintaining an active community and ensuring a live network is always operational, the foundation facilitates real-time monitoring. This approach empowers developers, allowing them to pinpoint any emerging issues promptly. Furthermore, it provides an ongoing platform to assess the efficiency, adaptability, and scalability of the solutions in place, ensuring that the network not only meets but exceeds the demands of its users.

\vspace{0.05cm}
\begin{center}
	\includegraphics[width=381pt]{roadmap}
\end{center}

\begin{itemize}
  \item Q2 2022 New JavaFX wallet
  \item Q1 2023 Hedgehog
  \item Q2 2023 Hedgehog live
  \item Q4 2023 Data storage
  \item Q1 2024 Hedgehog live
  \item Q2-Q3 2024 CPU compute
  \item Q4 2024 GPU compute
\end{itemize}

\noindent The following milestones are some secondary goals we are planning or already working on in order to demonstrate the liability of the network.
\begin{itemize}
  \item Technology demonstration of "Hello world" running on the network
  \item Compute work (GPU)
  \item Technology demonstration of an FTP server running on the network
  \item Domain name service
  \item Technology demonstration of SETI@Home running on the network
\end{itemize}

\noindent Rather than being a list of tasks, the above milestones is a list of goals that the foundation wants to achieve. While the list may still change, the foundation plans to target these goals.

As part of its expanding ecosystem, Unigrid is actively working on crafting a suite of web frontends to provide users seamless access to network storage and its myriad functionalities. Among these innovations is the Unigrid Drive, a cutting-edge product akin to Google Drive, designed to empower end-users to effortlessly store their data within the Unigrid network.

\section{Press Coverage \& Social Change}
The Unigrid Foundation recently won a court case in Sweden against the county of Västra Götaland, which is responsible for overseeing the registration of foundations. The crux of the matter revolved around the classification of cryptocurrencies as assets for foundations. Specifically, the county argued that merely separating crypto assets for a foundation did not fulfill the requisite duration requirement for the foundation's purpose. However, the Unigrid Foundation contended and successfully established in court that cryptocurrencies should be recognized as an accepted asset class during the formation of foundations~\cite{decentralized-internet,linkedin-paulk,linkedin-valuexag}.

This victory is seen as a major accomplishment not only for the Unigrid Foundation but also in paving the way for the future registration of foundations, potentially altering how crypto assets are perceived and handled in legal and foundational settings in Sweden and possibly beyond~\cite{linkedin-valuexag}.

\begin{center}
\vspace{0.1cm}
\href{https://www.cloudcomputing-insider.de/unigrid-plant-europaweite-cloud-auf-blockchain-basis-a-8514f7f959be74ec0a64e77ea064a007/}{\includegraphics[width=90pt]{logo-cci}}
\hspace{1cm}
\href{https://cointelegraph.com/news/how-a-blockchain-network-plans-to-tackle-the-internets-major-shortcomings}{\includegraphics[width=90pt]{logo-ct}}
\hspace{1cm}
\href{https://cryptodaily.co.uk/2021/11/unigrid-next-internet-revolution}{\includegraphics[width=90pt]{logo-cryptodaily}}
\vspace{0.1cm}
\end{center}

\noindent Unigrid, aiming to disrupt the traditional internet infrastructure by leveraging blockchain technology, has been covered by various media outlets. In an article by Cointelegraph, it is mentioned that Unigrid seeks to broaden internet access by reducing costs through blockchain, targeting a substantial market share in cloud computing~\cite{cointelegraph}. CryptoDaily has covered Unigrid's vision for an internet without borders, emphasizing privacy protection~\cite{cryptodaily1}. Moreover, another piece in CryptoDaily highlighted a significant investment commitment, underscoring the transformative power of Unigrid's decentralized storage offering~\cite{cryptodaily2}. In Germany, CloudComputing-Insider reported Unigrid's ambition to connect 10,000 European data centers using blockchain technology, offering a decentralized cloud solution~\cite{germanpress}.

\begin{center}
\vspace{0.35cm}
\href{https://news.bitcoin.com/unigrid-secures-25m-investment-commitment-from-gem-digital-partners-wesendit-targets-cloud-giants/}{\includegraphics[width=90pt]{logo-bitcoin}}
\vspace{0.3cm}
\end{center}

\noindent Unigrid has successfully secured a substantial investment commitment of \$25 million from GEM Digital Limited, marking a significant milestone in its journey. This financial backing is anticipated to catalyze Unigrid's market positioning and foster new strategic partnerships, notably with WeSendit, a pioneering file transfer service specializing in anonymous data transfer and decentralized storage solutions. The investment from GEM Digital Limited is not only a testament to Unigrid's innovative approach in leveraging blockchain technology to develop a decentralized and anonymous Internet but also a strong endorsement of its potential to revolutionize the internet infrastructure~\cite{bitcoincom,blockchair,problockchain}.



\section{Conclusion}
Since the advent of the Internet in the 1960's \cite{int1997} the Internet can be defined as a collection of nodes (computers) communicating with each other. The Unigrid Foundation aims to decentralize and load-balance these nodes, allowing anyone to partake in being a host node on the Internet. From the first contact with the network, your data is encrypted and sharded, making it impossible for anyone else to access this data without a key. Browsing across the network is also anonymous - securing your privacy rights.

\vspace{0.1cm}
\renewcommand{\arraystretch}{1.6}%
\begin{flushleft}
	\hypersetup{colorlinks = true, urlcolor = black, citecolor = black, linkcolor = black}
	\center \small
	\begin{tabular}{lrrrrr}
		\rowcolor{orange}\multicolumn{6}{c}{\color{black} \textbf{Worldwide Cloud Service Revenue Forecast \cite{gartner2019} (Billions of U.S. Dollars)}} \\
		\rowcolor{orange} & \color{black}2020 & \color{black}2021 & \color{black}2022 & \color{black}2023 & \color{black}2024 \\
		Business Process Services (BPaaS) \hspace{2.3cm}              &  53.1 &  57.0 &  59.9 & 65.2 & 71 \\
		\rowcolor{headerbgl} Application Infrastructure Services (PaaS) & 23.0 &  27.5 &  112 & 139 & 170.4 \\
		Application Services (SaaS)                                 & 110.5 & 126.7 & 167.3 & 197.3 & 232.3 \\
		\rowcolor{headerbgl} Management and Security Services       &  14.1 &  16.0 &  34.5 & 42.4 & 51.9 \\
		System Infrastructure Services (IaaS)                       &  49.1 &  61.9 &  114.8 & 150.3 & 195.4 \\
		\rowcolor{headerbgl} Total Market                           & 249.8 & 289.1 & 490 & 597.3 & 774.6
	\end{tabular}
\end{flushleft}

\vspace{0.6cm}
\noindent Gridnodes serve as the foundational pillars of the network, driving its functionality and ensuring seamless operations. Their critical role in the system comes with a well-deserved reward for the services they provide.

The global cloud service landscape has been experiencing a transformative growth phase. Based on the data from research firm Gartner, the worldwide cloud service revenue is set to grow significantly across various sectors. A closer look reveals the following insights:

\begin{itemize}
    \item Business Process Services (BPaaS) revenue, which was at \$53.1 billion in 2020, is projected to reach \$71 billion by 2024.
    \item Application Infrastructure Services (PaaS) has seen a surge from \$23 billion in 2020 to an anticipated \$170.4 billion in 2024.
    \item Application Services (SaaS) too is on an upward trajectory, expected to grow from \$110.5 billion in 2020 to \$232.3 billion by 2024.
    \item Management and Security Services are not far behind, with projections indicating a rise from \$14.1 billion in 2020 to \$51.9 billion in 2024.
    \item However, the most significant growth can be observed in the System Infrastructure Services (IaaS) sector. From \$49.1 billion in 2020, it is forecasted to reach a staggering \$195.4 billion by 2024.
    \item Overall, the total market is anticipated to expand from \$249.8 billion in 2020 to \$774.6 billion by 2024.
\end{itemize}

Given the substantial growth in the System Infrastructure Services (IaaS) sector — from \$38.9 billion in 2019 to a projected \$195.4 billion by 2024 — it's evident that the potential market share for the Unigrid network is vast. Unigrid is uniquely positioned to offer unmatched benefits, functionalities, and services that currently have no parallels in the market.


\section{Closing Thoughts}
This paper covers some of the features of the Unigrid network and why the foundation thinks they are essential for a future Internet that is healthy, stable and scalable. Many of the topics covered only skim the surface of and do not explain many of the intricate details of the underlying implementations. For additional information and for further details on how the network functions on a theoretical level, the base technical Unigrid white paper \cite{wp2021} broadly describes how the foundation plans to implement much of the functionality discussed in this orange paper.

\newpage
\begin{thebibliography}{999}
\bibitem{wp2021}
    Adam Waldenberg, The Unigrid Foundation,
    \emph{"Unigrid: A foundation for a decentralized, consensus-driven, segmented, blockchain-based Internet"},
    \href{https://www.unigrid.org/about}{www.unigrid.org},
    2021.

\bibitem{int1997}
    Barry M. Leiner, Vinton G. Cerf, David D. Clark, Robert E. Kahn, Leonard Kleinrock, Daniel C. Lynch, Jon Postel, Larry G. Roberts, Stephen Wolff,
    \emph{"Origins of the Internet"},
    \href{https://www.internetsociety.org/internet/history-internet/brief-history-internet}{www.internetsociety.org},
    1997.

\bibitem{fort2021}
    Fortune Business Insights,
    \emph{"Cloud Storage Market Size"},
    \href{https://www.fortunebusinessinsights.com/cloud-storage-market-102773}{www.fortunebusinessinsights.com},
    2021.

\bibitem{gartner2019}
    Gartner, Gartner,
    \emph{"Gartner Forecasts Worldwide Public Cloud Revenue to Grow 17.5 Percent in 2019"},
    \href{https://www.gartner.com/en/newsroom/press-releases/2019-04-02-gartner-forecasts-worldwide-public-cloud-revenue-to-g}{www.gartner.com},
    2019.

\bibitem{john2018}
    John Levine, CircleID,
    \emph{"How Big Is The Domain Business"},
    \href{https://www.circleid.com/posts/20180813_how_big_is_the_domain_business/}{www.circleid.com},
    2018.

\bibitem{jeb2019}
    Jeb Su, Forbes,
    \emph{"Amazon Owns Nearly Half Of The Public-Cloud Infrastructure Market Worth Over \$32 Billion: Report"},
    \href{https://www.forbes.com/sites/jeanbaptiste/2019/08/02/amazon-owns-nearly-half-of-the-public-cloud-infrastructure-market-worth-over-32-billion-report/ }{www.forbes.com},
    2019.

\bibitem{young2021}
    Joseph Young, Forbes,
    \emph{"NFT Market Rages On: NFTs Market Cap Grow 1,785\% In 2021 As Demand Explodes"},
    \href{https://www.forbes.com/sites/youngjoseph/2021/03/29/nft-market-rages-on-nfts-market-cap-grow-1785-in-2021-as-demand-explodes/
}{www.gartner.com},
    2021.

\bibitem{mark2019}
    Mark Drake ,Digital Ocean,
    \emph{"Understanding Database Sharding"},
    \href{https://www.digitalocean.com/community/tutorials/understanding-database-sharding}{www.digitalocean.com},
    2019.

\bibitem{jain2021}
    Shruti Jain ,Deloitte,
    \emph{"Can blockchain accelerate Internet of Things (IoT) adoption?"},
    \href{https://www2.deloitte.com/ch/en/pages/innovation/articles/blockchain-accelerate-iot-adoption.html}{www.deloitte.com},
    2021.

\bibitem{wiki2021}
    Wikipedia,
    \emph{"Internet censorship in China"},
    \href{https://www.wikipedia.org/wiki/Internet_censorship_in_China}{www.wikipedia.org},
    2021.

\bibitem{valuex2023}
	ValueX,
	\emph{We empower projects to become successful businesses},
	\href{https://www.valuex.at}{www.valuex.at},
	2023.

\bibitem{nfmedia2023}
	NFMedia,
	\emph{Invest in your Future},
	\href{https://www.nfmedia.org}{www.nfmedia.org},
	2023.

\bibitem{decentralized-internet}
	Decentralized Internet,
	\emph{The Unigrid Foundation wins court decision},
	2023,
	\url{https://www.decentralized-internet.com/unigrid-foundation-wins-court-decision}.

\bibitem{linkedin-paulk}
	Paul K.,
	\emph{The Unigrid Foundation wins court decision},
	2023,
	\url{https://www.linkedin.com/pulse/unigrid-foundation-wins-court-decision-paul-k}.

\bibitem{linkedin-valuexag}
	Valuex AG,
	\emph{The Unigrid Foundation wins court decision},
	2023,
	\url{https://www.linkedin.com/posts/valuex-ag_unigrid-foundation-wins-court-decision}.


\bibitem{cointelegraph}
	Cointelegraph,
	\emph{How a blockchain network plans to tackle the internet's challenges},
	\url{https://cointelegraph.com/news/how-a-blockchain-network-plans-to-tackle-the-internet-s-challenges}.

\bibitem{cryptodaily1}
	CryptoDaily,
	\emph{What Makes Unigrid "The Next Internet Revolution"?},
	\url{https://cryptodaily.co.uk/2023/07/what-makes-unigrid-the-next-internet-revolution}.

\bibitem{cryptodaily2}
	CryptoDaily,
	\emph{Unigrid Secures \$25M Investment Commitment from GEM Digital, Partners with WeSendit},
	\url{https://cryptodaily.co.uk/2023/08/unigrid-secures-25m-investment}.

\bibitem{germanpress}
	CloudComputing-Insider,
	\emph{Unigrid plant europaweite Cloud auf Blockchain-Basis},
	\url{https://www.cloudcomputing-insider.de/unigrid-plant-europaweite-cloud-auf-blockchain-basis-a-1041942}.

\bibitem{bitcoincom}
	news.bitcoin.com,
	\emph{Unigrid Secures \$25M Investment Commitment from GEM Digital},
	\url{https://news.bitcoin.com/unigrid-secures-25m-investment-commitment-from-gem-digital\\-partners-with-wesendit}.

\bibitem{blockchair}
	blockchair.com,
	\emph{Unigrid Secures \$25M Investment Commitment from GEM Digital},
	\url{https://blockchair.com/press-releases/unigrid-secures-25m-investment-commitment-from-gem-digital\\-partners-with-wesendit}.

\bibitem{problockchain}
	pro-blockchain.com,
	\emph{Unigrid Secures \$25M Investment Commitment from GEM Digital},
	\url{https://pro-blockchain.com/unigrid-secures-25m-investment-commitment-from-gem-digital\\-partners-with-wesendit}.

\end{thebibliography}

\end{document}

\end{thebibliography}
\end{document}